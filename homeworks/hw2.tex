\documentclass[11pt]{article}
\usepackage{fullpage,url,hyperref}
\usepackage{amsmath}
\usepackage{amssymb}
\usepackage[shortlabels]{enumitem}

\usepackage[letterpaper,top=1in,bottom=1in,left=1in,right=1in,nohead]{geometry}

\setlength{\parindent}{0in}
\setlength{\parskip}{6pt}

\begin{document}
\thispagestyle{empty}
{\large{\bf ECE / CS 6501: Geometry of Data \hfill Due Tu 10/15}}\\

{\LARGE{\bf Homework 2: Geodesics and Manifold Statistics}}
\vspace{0.2\baselineskip}
\hrule

{\bf Instructions:} Submit a single Jupyter notebook (.ipynb) of your work to
Collab by 11:59pm on the due date. All code should be written in Python. {\bf Be
  sure to show all the work involved in deriving your answers! If you just give
  a final answer without explanation, you may not receive credit for that
  question.}

You may discuss the concepts with your classmates, but write up the answers
entirely on your own. Do not look at another student's answers, do not use
answers from the internet or other sources, and do not show your answers to
anyone.

\section{Geodesic Equation on $S^2$}
\begin{enumerate}[(a)]
\item Consider the standard $(\theta, \phi)$ parameterization of the sphere:
  \begin{align*}
    s &: (0, 2 \pi) \times \left(-\frac{\pi}{2}, \frac{\pi}{2}\right) \rightarrow \mathbb{R}^3\\
    s(\theta, \phi) &= (\cos \theta \cos \phi, \sin \theta \cos \phi, \sin \phi).
  \end{align*}
  Write down equations for the coordinate tangent vectors
  $\frac{\partial s}{\partial \theta}$ and $\frac{\partial s}{\partial
    \phi}$. These should be vectors in $\mathbb{R}^3$. {\bf Hint:} We did these
  on the board in class already!

\item Using the induced metric of $\mathbb{R}^3$, write down the coefficients of
  the Riemannian metric, $g$, on $S^2$ in terms of the coordinates
  $\theta, \phi$. {\bf Hint:} You should end up with a $2 \times 2$ matrix,
  whose entries are just Euclidean $\mathbb{R}^3$ inner products of the tangent
  vectors from the previous part.

\item Compute the inverse metric $g^{-1}$. Again, this will be a $2 \times 2$
  matrix whose entries are functions of $\theta, \phi$.

\item Compute formulas for the Christoffel symbols, $\Gamma_{ij}^k$.

\item Show that the equator circle $\gamma(t) = s(t, 0)$ is a geodesic.

\item Provide an argument that all great circles (circles on $S^2$ with radius =
  1) are geodesics. You don't need a lengthy or precise proof, just a brief and
  informal argument will suffice.
  
\end{enumerate}

\section{Statistics on Shape Manifolds}
In this project you will implement Kendall's shape space and simple shape
statistics, namely, the Fr\'{e}chet mean. You first must decide on a data
structure for representing a 2D point set (called an ``object'' throughout this
assignment).  For an object containing $n$ points, you may decide to use either
an $2 \times n$ real matrix or a list of $n$ complex numbers. Also, you will
want to have a way to plot your resulting shapes.

\paragraph{Methods:} You will need to implement functions that perform the
following operations:

\begin{enumerate}
\item Project an object onto the preshape sphere by removing its centroid
and scaling it to unit norm. This is a preprocessing step that all objects
will go through before applying any of the following routines.

\item Perform Ordinary (a.k.a. Orthogonal) Procrustes Analysis (OPA). This
should input a target preshape and moving preshape and output the aligned
version of the moving preshape.

\item Compute the exponential map on Kendall shape space, i.e., given a preshape
and tangent vector, compute the preshape at the end of the corresponding
geodesic.

\item Compute the log map on Kendall shape space, i.e., given two preshapes,
compute the tangent vector at the first preshape that is the initial condition
of the geodesic segment between the two.

\item Estimate the Fr\'{e}chet mean shape using gradient descent as described in
  class.

\item Compute the approximate principal geodesic analysis (PGA), that is,
perform principal component analysis (PCA) in the tangent space to the
Fr\'{e}chet mean.
\end{enumerate}

\paragraph{Experiments:} You will test your shape analyis routines using two
datasets: one synthetic and one from real images.

\begin{enumerate}
\item Random triangles. Generate a set of 100 triangles with the following three
points:
$$p_0 = (-1,0), \quad p_1 = (1,0), \quad p_2 = (0,s),$$
where $s$ is a Gaussian random variable with mean 1 and standard deviation 0.5.
Next, add zero-mean Gaussian noise with s.d. 0.1 to all six coordinates of each
of the 100 triangles.

\item Corpus callosum shapes:
\url{https://tomfletcher.github.io/GeometryOfData/homeworks/cc-shapes.zip}

Each corpus callosum object consists of 64 points listed in an ASCII text file.
Each line of the file contains a single point ($x$ and $y$ coordinates).
\end{enumerate}

\paragraph{Report:} You should discuss your work in your Jupyter notebook. Be
sure to include the following:
\begin{itemize}
\item Describe your implementation. You do not need to recount the theory and
equations behind all of the methods, just a brief description of how you
implemented them and the issues you faced.
\item Describe both the triangle and corpus callosum experiments. For each
experiment include:
\begin{itemize}
\item Pick two objects from the data set and plot a geodesic between them (this
should be a sequence of objects deforming from one to the other).
\item Plots of 1) the raw object data (i.e., all points from all objects
overlaid in one plot), 2) the preshapes aligned to the Fr\'{e}chet mean, and
3) the Fr\'{e}chet mean shape (in a different color). Can you see an obvious
difference in the aligned shapes?
\item A plot of the eigenvalues for each mode (scree plot). Looking at this
plot, how many modes would you use to describe the data?
\item Plot the modes of variation at $0, \pm 1, \pm 2$ standard deviations. You
only need to plot up to 3 different modes (maybe fewer, if you decided in the
previous question that fewer were needed to represent the data).
\item Describe what you are seeing in the modes of variation. For the triangle
case, is this the variation you expected?
\end{itemize}
\item Discuss: What is the maximal number of modes that are possible for
triangle shapes? In general, what are the maximal number of modes that are
possible for a set of objects with $n$ points?
\end{itemize}
\end{document}
